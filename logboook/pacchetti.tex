% ================================================================
%                          DOCUMENTO
% ================================================================

%\documentclass{scrbook}
%\documentclass[openany]{report}
\documentclass[a4paper, 11pt, oneside, openary]{book} % A4 paper size, default 11pt font size and oneside for equal margins
%\documentclass{memoir}

\raggedbottom %evita che LateX aggiunga da solo dello sèpazio aggiuntivo per ragioni tipografiche

% ================================================================
%                          PACCHETTI
% ================================================================

\usepackage[utf8]{inputenc}            % lettere accentate da tastiera
\usepackage[T1]{fontenc}               % codifica dei font
\usepackage[english, italian]{babel}   % lingua del documento
\usepackage{url}                       % per scrivere gli indirizzi Internet
\usepackage{lipsum}                    % genera testo fittizio
\usepackage{marginnote}                % per le note a margine (es date)
\usepackage[marginparsep=0.8cm]{geometry}                  % per decidere le dimensioni e margini
\usepackage[most]{tcolorbox}           % questo per i riquadri
%\geometry{marginparsep=5cm}
%\geometry{a4paper,top=3cm,bottom=3cm,left=3.5cm,right=3.5cm, heightrounded,bindingoffset=5mm}
\usepackage{hyperref}                  % link
\usepackage{float}                     % posizione figure
\usepackage{wrapfig}

% dare una spiegazione
\usepackage{tikz}
\usetikzlibrary{arrows}

% ??
\usepackage{amsmath,amssymb,amsthm,thmtools}

% glossario
\usepackage{makeidx}

% ??
\usepackage{mathtools} % per \MoveEqLeft

% QR-code
\usepackage{qrcode}

\usepackage{array} % per formattare le colonne di un tabular

% disegni (non ho idea di cosa sia questa roba)
\usepackage{tikz}
\usetikzlibrary{calc,angles,positioning,intersections,quotes,decorations.markings}
\usetikzlibrary{cd,calc,backgrounds} % commutative diagrams
%\usepackage{tkz-euclide}
%\usetkzobj{all}


%\usepackage[svgnames]{xcolor} % Required for colour specification

%\usepackage{fouriernc} % Use the New Century Schoolbook font

\usepackage{ulem} %questo per far sì che funzionino testi barrati e/o sottolineati

\usepackage{collectbox}
\usepackage{graphicx}% just for the example text

\definecolor{airforceblue}{rgb}{0.36, 0.54, 0.66}

\hypersetup %crea i collegamenti ipertestuali all’interno del documento, rendendo cliccabili i riferimenti incrociati [DEVE ESSERE L'ULTIMO FRA I PACCHETTI INSERITI]