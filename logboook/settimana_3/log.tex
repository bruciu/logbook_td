















\LogMark{I prof non sentono}{9:00}
E noi possiamo dire quello che ci pare. Non abbiamo installato TINA sigh. Non ho fatto in tempo a scrivere il link, non è che lo puoi scrivere qui? Ok niente, vedo da elearning. Segui il mio cursore? Cattivo. Ma io ho LTspice ahah iNFATTI non mi risulta affatto friendly


\NewSection{Buongiorno!!}
\begin{verbatim}
|\---/|
| o_o |
 \_^_/
 
      /\_/\ /\_/\
     (^ ^) {@ @}
     ==~== ==o==
      \@/   \^/
      |=|   ###_
(    /  \  /    \
 \  /   |  |     \
  )/ ||||  ||||(  \   \
 (( /||||  |||| \  )   )
   m !m!m  m!m! m-~(__/
   https://www.asciiart.eu/animals/cats
   Art by Joan Stark

           .'\   /`.
         .'.-.`-'.-.`.
    ..._:   .-. .-.   :_...
  .'    '-.(o ) (o ).-'    `.
 :  _    _ _`~(_)~`_ _    _  :
:  /:   ' .-=_   _=-. `   ;\  :
:   :|-.._  '     `  _..-|:   :
 :   `:| |`:-:-.-:-:'| |:'   :
  `.   `.| | | | | | |.'   .'
    `.   `-:_| | |_:-'   .'
 jgs  `-._   ````    _.-'
          ``-------''

\end{verbatim}
\LogMark{Analisi}{9:49}
Usiamo gli strumenti di analisi DC...
Ci sono vari strumenti sotto Analysis->DC Analysis, proviamo le varie possibilità per analizzare correnti e tensioni sui vari componenti/nodi.
\begin{enumerate}
    \item xxx: sonda
    \item xxx: tavola
    \item xxx: grafico in funzione di un parametro
    \item xxx: in funzione della temperatura (non ci serve)
\end{enumerate}
analisy transfer control su rx per vedere quando si azzera, si azzera a 2k, come ci aspettavamo

\Nota{
    Posso usare l'Help relativo ai componenti per ottenere informazioni dettagliate su di essi.
}

I breve cosa stiamo facendo:
\begin{enumerate}
    \item clicchiamo new line script. Una volta scritto il programma qui, puoi salvarlo anche in pdf o LateX. n.b. tutte le cose bnumeriche si possono fare in simbolico. Le variabili vanno dichiarate. solve() per risolvere le equazioni algebriche. lezione1.pdf
\end{enumerate}

Cosa sto scrivendo su Matlab:
syms x
int(sqrt(tan(x)))
pretty(ans)
latex(ans)
 
output dell'istruzione LateX:
\begin{equation}
ans = '\frac{\sqrt{2}\,\left(\ln\left(\sqrt{2}\,\sqrt{\mathrm{tan}\left(x\right)}-\mathrm{tan}\left(x\right)-1\right)-\ln\left(\mathrm{tan}\left(x\right)+\sqrt{2}\,\sqrt{\mathrm{tan}\left(x\right)}+1\right)\right)}{4}+\frac{\sqrt{2}\,\left(\mathrm{atan}\left(\sqrt{2}\,\sqrt{\mathrm{tan}\left(x\right)}-1\right)+\mathrm{atan}\left(\sqrt{2}\,\sqrt{\mathrm{tan}\left(x\right)}+1\right)\right)}{2}'
\end{equation}

\LogMark{creiamo un Live Script}{10:09}
Ora lavoreremo in un Live Script così da poter vedere tutta la storia.
\Nota{
    I comabdi %\verb|help symbolic| e \verb| doc symbolic|
}
VPA per usare un numero arbitrario di cifre significative
\Avvertimento{
    Attento a l'ordine delle operazioni ...
}

\LogMark{Esplorazione comandi simbolici}{10:15-...}
semplificazione, espansione, soluzione equazioni, pretty, latex, int, diff, grafici, dsolve vari

\Nota{
    con latex() ci possiamo far dare il codice latex di un simbolico, possiamo anche esportare in vari formati tutto il documento.
}

\LogMark{Ponte di Wheatstone}{10:34}
Introduzione all'esercizio del ponte di Wheatstone
\TODO{
    Da fare a casa (vedi slides)
}
%Si costruisce una matrice che deriva dal circuito: A * x = b, dove il vettore x è incognito
