% sono presi da paolini
% https://github.com/paolini/AnalisiUno/blob/master/AnalisiUno.tex

\newcommand{\eps}{\varepsilon}
\renewcommand{\phi}{\varphi}
\newcommand{\loc}{\mathit{loc}}
\newcommand{\weakto}{\rightharpoonup}
\newcommand{\implied}{\Longleftarrow}
\let\subsetstrict\subset
\renewcommand{\subset}{\subseteq}
\renewcommand{\supset}{\supseteq}

% calligraphic letters
\newcommand{\A}{\mathcal A}
\newcommand{\B}{\mathcal B}
\newcommand{\C}{\mathcal C}
\newcommand{\D}{\mathcal D}
\newcommand{\E}{\mathcal E}
\newcommand{\F}{\mathcal F}
\newcommand{\FL}{\mathcal F\!\mathcal L}
\renewcommand{\H}{\mathcal H}
\newcommand{\K}{\mathcal K}
\renewcommand{\L}{\mathcal L}
\newcommand{\M}{\mathcal M}
\renewcommand{\P}{\mathcal P}
\renewcommand{\S}{\mathcal S}
\newcommand{\U}{\mathcal U} %% intorni

% blackboard letters
\newcommand{\CC}{\mathbb C}
\newcommand{\HH}{\mathbb H}
\newcommand{\KK}{\mathbb K}
\newcommand{\NN}{\mathbb N}
\newcommand{\QQ}{\mathbb Q}
\newcommand{\RR}{\mathbb R}
\newcommand{\TT}{\mathbb T}
\newcommand{\ZZ}{\mathbb Z}

\newcommand{\abs}[1]{{\left|#1\right|}}
\newcommand{\Abs}[1]{{\left\Vert #1\right\Vert}}
\newcommand{\enclose}[1]{{\left( #1 \right)}}
\newcommand{\Enclose}[1]{{\left[ #1 \right]}}
\newcommand{\ENCLOSE}[1]{{\left\{ #1 \right\}}}
\newcommand{\floor}[1]{\left\lfloor #1 \right\rfloor}
\newcommand{\ceil}[1]{\left\lceil #1 \right\rceil}

\newcommand{\To}{\rightrightarrows}
\renewcommand{\vec}[1]{\boldsymbol #1}
\newcommand{\defeq}{:=}
\DeclareMathOperator{\divergence}{div}
\renewcommand{\div}{\divergence}
% \DeclareMathOperator{\ker}{ker}  %% already defined
\DeclareMathOperator{\Imaginarypart}{Im}
\renewcommand{\Im}{\Imaginarypart}
\DeclareMathOperator{\Realpart}{Re}
\renewcommand{\Re}{\Realpart}
%\DeclareMathOperator{\arg}{arg}
\DeclareMathOperator{\tg}{tg}
\DeclareMathOperator{\arctg}{arctg}
\DeclareMathOperator{\tgh}{tgh}
\DeclareMathOperator{\settsinh}{settsinh}
\DeclareMathOperator{\settcosh}{settcosh}
\DeclareMathOperator{\setttgh}{setttgh}
\DeclareMathOperator{\tr}{tr}
\DeclareMathOperator{\im}{im}
\DeclareMathOperator{\sgn}{sgn}
\DeclareMathOperator{\diag}{diag}


% ================================================================
%                         NOSTRI COMANDI
% ================================================================

% cambia il nome del titolo (non serve piò)
\addto\captionsitalian{
   \renewcommand\chaptername{LogBook n.} 
}

\newcommand{\giornocorrente}{00/00/00}

% crea una linea di separazione (es: \LogMark{Nuova parte}{24/09/2020}{10:10})
\newcommand{\LogMark}[2]
{
    %\vspace{0.5cm}
    \subsection*{#1}
    \vspace{-0.3cm}

    \reversemarginpar
    \marginnote
    {
        \textcolor{gray}{\giornocorrente} \\
        \textcolor{gray}{#2}
    }%[0.3cm]
    \normalmarginpar
    
    % disegna una linea orizzontale
    \noindent\rule{\textwidth}{0.5pt}
}

\newcommand{\NewDay}[2]
{
    \renewcommand{\giornocorrente}{#2}

	\section*{#1 \qquad #2}
	
	\vspace{-0.5cm}
	
	\noindent\textcolor{red}{\rule{\textwidth}{1pt}} % Thick horizontal rule
	
	\vspace{2pt}\vspace{-\baselineskip} % Whitespace between rules
	
	\noindent\textcolor{red}{\rule{\textwidth}{0.4pt}} % Thin horizontal rule
	
	%\vspace{-0.5cm}
}

\newcommand{\Orario}[1]
{
    \reversemarginpar
    \marginnote
    {
        \textcolor{gray}{#1}
    }
    \normalmarginpar
}

\newcommand{\NewSection}[1]
{
    \vspace{1cm}
    \section{#1}
}

% ================================
%          RIQUADRI
% ================================
%textmarker style from colorbox doc
\tcbset{textmarker/.style={%
        enhanced,
        parbox=false,boxrule=0mm,boxsep=0mm,arc=0mm,
        outer arc=0mm,left=6mm,right=3mm,top=7pt,bottom=7pt,
        toptitle=1mm,bottomtitle=1mm,oversize}}
% define new colorboxes
\newtcolorbox{hintBox}{textmarker,
    borderline west={6pt}{0pt}{yellow},
    colback=yellow!10!white}
\newtcolorbox{importantBox}{textmarker,
    borderline west={6pt}{0pt}{red},
    colback=red!10!white}
\newtcolorbox{noteBox}{textmarker,
    borderline west={6pt}{0pt}{green},
    colback=green!10!white}
\newtcolorbox{todoBox}{textmarker,
    borderline west={6pt}{0pt}{violet},
    colback=violet!10!white}

% define commands for easy access

% NOTA
\newcommand{\Nota}[1]{\begin{noteBox} \textbf{Nota:\quad} #1 \end{noteBox}}

% AVVERTIMENTO
\newcommand{\Avvertimento}[1]{\begin{hintBox} \textbf{Avvertimento:\quad} #1 \end{hintBox}}

% IMPOTANTE
\newcommand{\Importante}[1]{\begin{importantBox} \textbf{Importante:\quad} #1 \end{importantBox}}

% TODO
\newcommand{\TODO}[1]{\begin{todoBox} \textbf{TODO:\quad} #1 \end{todoBox}}